\documentclass[a4paper,10pt, notitlepage]{report}
\usepackage[utf8]{inputenc}
\usepackage{natbib}
\usepackage{amssymb}
\usepackage{amsmath}
\usepackage{undertilde}
\usepackage{enumitem}
\usepackage[portuguese]{babel}


% Title Page
\title{Trabalho II: o algoritmo EM.}
\author{Disciplina: Inferência Estatística \\ Aluno: Iara Cristina Mescua Castro \\ Curso: Matemática Aplicada}

\begin{document}
	\maketitle
	
	\textbf{Data de Entrega: 28 de Setembro de 2022.}
	
	\section*{Questões}
	\begin{enumerate}
		\item Defina claramente todos os passos do algoritmo EM (faça um glossário de termos se precisar);
		\item \textbf{Um exemplo motivador:} Suponha que temos duas moedas, Moeda $1$ e Moeda $2$ de modo que $\operatorname{Pr}(\text{Cara} \mid \text{Moeda} = 1) = p_1$ e $\operatorname{Pr}(\text{Cara} \mid \text{Moeda} = 2) = p_2$; 
		Suponha agora que fazemos o seguinte experimento:
		\begin{itemize}
			\item[(i)] Selecionamos uma moeda aleatóriamente com probabilidade $1/2$;
			\item[(ii)] Lançamos a moeda selecionada $m$ vezes;
			\item[(iii)] Repetimos (i) e (ii) $n$ vezes.
		\end{itemize}
		Podemos representar os dados advindos deste experimento como 
		\begin{center}
			\begin{tabular}{ c c c c c}
				$X_{11}$ & $\ldots$ & $X_{1m}$ & & $M_1$  \\ 
				$X_{21}$ & $\ldots$ & $X_{2m}$ & & $M_2$  \\
				$\vdots$ & $\ldots$ & $\vdots$ & & $\vdots$  \\
				$X_{n1}$ & $\ldots$ & $X_{nm}$ & & $M_n$ ¨
			\end{tabular}
		\end{center}
		
		onde os $X_{ij}$ são variáveis Bernoulli que guardam o resultados do lançamento da moeda  e $M_i \in \{ 1, 2\}$ é a variável aleatória que guarda qual moeda foi selecionada na $i$-ésima rodada do experimento.
		
		\textbf{Desenvolva} um esquema EM para obter o EMV para $\theta = (p_1, p_2)$ quando desconhecemos os valores de $M_i$, isto é, quando não sabemos que moeda foi escolhida em cada uma das $n$ rodadas.
		
		\item Mostre que a sequência $\theta^{(j)}$ é monotônica e não descrescente com respeito à verossimilhança, isto é,
		\[ L\left(\theta^{(j + 1)}\right) \geq L\left(\theta^{(j)}\right) \]
		Dica (fortemente aconselhado): ver exercício 7.31 de Casella \& Berger (2002). 
		\item Discuta a importância do método EM: quando ele é aplicável? Vale sempre a pena? O que o item anterior demonstra sobre o método?
	\end{enumerate}
	\newpage
	\section*{Respostas}
	
	\begin{enumerate}
		\item Do ponto de vista frequentista, pode-se pensar que os dados observados são gerados a partir de variáveis aleatórias X conjuntamente com dados faltantes ou não observados Z. \\
		$Y=(X,Z)$ representa o vetor de dados completos.
		
		Dado os dados observados, x, deseja-se maximizar a função de verossimilhança $L(\theta, x)$. Frequentemente isto não é simples no entanto pode-se simplificar usando as densidades $Y|\theta$ e $Z|x,\theta$. O algoritmo EM faz isso.\\
		
		Seja $f_x(x|\theta)$ e $f_y(y|\theta)$ as distribuições dos dados observados e os dados completos, respectivamente\\
		
		$f_x(x|\theta) = \int f(y|\theta) dz$ e $f_{z|x}(z|x,\theta) = \frac{f_y (y|\theta)}{f_x(x|\theta)}$... (A)\\
		$$L(Y|\theta) = L(X,Z|\theta)$$\\ O objetivo do algoritmo EM e encontrar o máximo de $L(Y|\theta)$ com relação a $\theta$. Seja $\theta^{(t)}$ o máximo obtido na iteração t. t = 0,1,2,... Define-se por $Q(\theta, \theta^{(0)})$
		
		$$Q(\theta, \theta^{(t))} = E\{log L(\theta, Y)\} \; \; (1)$$
		$$= E\{\log f_y(y|\theta)|x,\theta^{(t)}\} \; \; (2) $$
		$$= \int [\log f_y(y|\theta)] \log f_{z|x}(z| x,\theta^{(t)}) dz \; \; (3) $$
		
		Onde (3) estabelece que z é a única parte aleatória dado X = x.
		
		O algoritmo EM é inicializado em $O^{(0)}$ e alterna nos seguites passos:
		
		1) $E$ STEP: Calcular $Q(\theta, \theta^{(0)})$.\\
		2) $M$	STEP: Maximizar o $Q(\theta, \theta^{(0)})$.\\
		3) Retornar a 1 até convergência.
		
		\item Seja $x_j$ o resultado da j-ésima sequência de tamanho "n"
		
		$$P(x_i) = \frac{1}{2} P(x_i|M_1) + \frac{1}{2} P(x_i|M_2), \; \; j=1,...$$
		
		Dada a moeda $M_j \rightarrow$ cada elemento da sequência segue uma distribuição de Bernoulli com probabilidade $p_1$ ou $p_2$.\\
		
		- O problema é encontrar o EMV de $\theta = (p_1, p_2)$.\\
		- Considere $Y = (Z, \underbar{X})$, onde:
		
		$$Y_1= (Z_1, \underbar{X}_1), Y_2 = (Z_2, \underbar{X}_2), ..., Y_n = (Z_n,\underbar{X}_n)$$
		
		$Z_j$ é um indicador que identifica de que a moeda é $M_1$ ou $M_2$.
		
		$$\rightarrow Z_j = 0 \Rightarrow M_1 \; ou \; Z_j = 1 \Rightarrow M_2 $$
		
		Logo $Z = (z_1, ..., z_n)$ corresponde a variável de alocação não observada. Assim a verossimilhança completa pode ser representada por:
		
		$$\log L(\theta, Y) = \sum_{j=1}^{n}[(1-z_j)\log P(\underbar{X}_i | M_1) + z_j \log P(\underbar{X}_i|M_2)]$$
		
		Logo:
		
		$$Q(\theta, \theta^{(t)}) = E\{\sum_{j=1}^{n}[(1-z_j)\log P(\underbar{X}_i | M_1) + z_j \log P(\underbar{X}_i|M_2)] | \utilde{x}, \theta^{(t)}\}$$
		
		$= E\{\sum_{j=1}^{n}(1-z_j)[\sum_{k=1}^{m}x_{jk} \log p_1 + m - \sum_{k=1}^{m} x_{jk} \log (1-p_1)] +
		 \sum_{j=1}^{n}z_j [\sum_{k=1}^{m}x_{jk} \log p_2 + m - \sum_{k=1}^{m} x_{jk} \log (1-p_2)]  | \utilde{x}, \theta^{(t)}\} ... \;\; (\alpha)$
		 
		 E-Step
		 
		 Logo: Maximizar $Q(\theta, \theta^{(t)})$ equivale a maximizar $(\alpha)'$ com relacão a $\theta = (p_1, p_2)$ temos que o valor que maximiza $\alpha$. E dado por:
		 
		 $$P_1^{(t+1)} = \frac{\sum_{j=1}^{n} (1 - E(Z_j | \utilde{x}, \theta^{(t)}) \sum_{k=1}^{m}x_{jk}}{m \sum_{j=1}^{n} )1 - E(Z_j | \utilde{x}, \theta^{(t)})}, \;\; (*)$$
		 
		 $$P_2^{(t+1)} = \frac{\sum_{j=1}^{n} E(Z_j | \utilde{x}, \theta^{(t)} \sum_{k=1}^{m}x_{jk}}{m \sum_{j=1}^{n} E(Z_j | \utilde{x}, \theta^{(t)})}, \;\; (**)$$
		 
		 Onde:
		 
		  $$E(Z_j | \underbar{x},\theta^{(t)}) = \frac{\frac{1}{2} P(\utilde{x_j} |M_1)}{\frac{1}{2} P(\utilde{x_j} |M_1) + \frac{1}{2} P(\utilde{x_j} |M_2)} \;\; (***)$$ 
		  
		  $$z_j^{(t+1)} = E(z_j | \utilde{x}, \theta^{(t)}) = \frac{\hat{p_2}^{\sum_{k=1}^{m}x_{jk}} (1 - \hat{p_2})^{m - \sum_{k=1}^{m}x_{jk}}}{\hat{p_1}^{\sum_{k=1}^{m}x_{jk}} (1 - \hat{p_1})^{m - \sum_{k=1}^{m}x_{jk}} + \hat{p_2}^{\sum_{k=1}^{m}x_{jk}} (1 - \hat{p_2})^{m - \sum_{k=1}^{m}x_{jk}}}$$
		  
		 Então o algoritmo EM:\\
		
		- Inicia ${P_1}^0$ e ${P_2}^0$
		
		E-Step: determina $(\alpha')$
		M-Step: encontra a solução $P_1^{(t+1)}$ e $P_2^{(t+1)}$ através de $(*)$, $(**)$, $(***)$.
		 
		\item De $(A)$ temos que:
		$$\log f_x(x|\theta) = \log f_y(y|\theta) - \log f_{z|x}(z|x,\theta)$$
		
		Assim: $$E\{\log f_x(x|\theta| x, \theta^{(t)})\} = E\{\log f_y(y|\theta) |x,\theta^{(t)}\} - E\{\log f_{z|x} (z|x,\theta)|x,\theta^{(t)}\} \; ...\; \; (4)$$
		Onde as esperanças são calculadas com relação a $z|x,\theta^{(t)}$.\\
		Logo, de $(4)$ temos que:
		
		$$\log f_x(x|\theta) = Q(\theta|\theta^{(t)}) - H(\theta|^{(t)}) \; ... \;\; (5)$$
		Onde: $$H(\theta|\theta^{(t)}) = E\{\log f_y(y|\theta)|x,\theta^{(t)}\}$$
		A importância de $1$ está relacionada a que $H(\theta|\theta^{(t)})$ é maximizada com relação a $\theta$ quando $\theta = \theta^{(t)}$
		
		$$H(\theta^{(t)}|\theta^{(t)}) - H(\theta|\theta^{(t)})$$
		$$= E\{\log f_{z|x}(z|x,\theta^{(t)}) | -\log f_{z|x}(z|x,\theta)\}$$
		$$= E\{-\log \left[\frac{f_{z|x}(z|x,\theta)}{f_{z|x,\theta^{(t)}}}\right] | x,\theta^{(t)}\}$$
		$$= \int -\log \left[\frac{f_{z|x}(z|x,\theta)}{f_{z|x,\theta^{(t)}}}\right] \times f_{z|x}(z|x,\theta^{(t)}) dz$$
		$$\geq -\log \int f_{z|x} (z|x,\theta^{(t)}) dz = 0$$
		
		O último resultado é obtido da desigualdade de Jensen, já que $-\log u$ é estritamente convexa.\\
		
		Se $\theta \neq \theta^{(t)} \rightarrow H(\theta|\theta^{(t)})$ é menor que $H(\theta^{(t)}|\theta^{(t)})$.
		Em particular se $O^{(t+1)}$ maximizar $Q(\theta|\theta^{(t)})$ com relação a $\theta$, 
		
		$$\log f_x(x|\theta^{(t+1)}) - \log f_x(x|\theta^{(t)}) \geq 0$$
		$$\Longrightarrow L(\theta^{(t+1)}) \geq L(\theta^{(t)})$$
		
		Já que Q cresce quando H decresce.

		$$\Longrightarrow Q(\theta^{(t+1)}|\theta^{(t)}) > Q(\theta^{(t)}|\theta^{(t)})$$
		\newpage
		\item Reforçando o que foi dito na Questão 1, o algoritmo é importante para maximizar a função de verossimilhança de parâmetros quando há dados faltantes ou ão observados. Ele também pode ser aplicado quando há data latente não observada  que nunca teve como objetivo iniciar ser observada. É amplamente usada em machine learning, especialmente em "data clustering". Também é usada em processamento de linguagem natural (PLN). Também pode ser usada para descobrir a densidade Gaussiana de uma função.
		
		Entretanto, nem sempre vale a pena, pois a convergência do algoritmo EM é muito lenta, pode convergir para apenas um "ótimo" local e leva em consideração a probabilidade para frente e para trás. O método de Newton-Raphson pode ser mais conveniente em algumas ocasiões por ter um tempo de convergência menor, mas apenas se torna uma aproximação precisa quando $\theta^{(t)}$ está próximo de $\hat{\theta}$, Então é recomendado usar o algoritmo EM para as primeiras iterações.
		
		É dito que $L(\theta^{(t)})$ é uma estimativa da quantidade $\left[\frac{p(x^{(i),z^{(i)}}); \theta)}{{Q_i(z^{(i)})}_i}\right]$ em respeito a $z^{(i)} \sim Q_i$. Queremos fazer a cota inferior com o mais próximo possível de $\theta$, então vamos fazer a desigualdade da Questão 3 manter igualdade em nosso valor particular de $\theta$. Ela prova que o algoritmo EM sempre cresce monotonamente, a cada iteração, com o log da função de verossimilhança.
		
	\end{enumerate}
	
	%\bibliographystyle{apalike}
	%\bibliography{refs}
	
\end{document}          