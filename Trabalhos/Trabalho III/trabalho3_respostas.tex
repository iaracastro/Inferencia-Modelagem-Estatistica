\documentclass[a4paper,10pt, notitlepage]{report}
\usepackage[utf8]{inputenc}
\usepackage[margin=1in]{geometry}
\usepackage{natbib}
\usepackage{amssymb}
\usepackage{amsmath}
\usepackage{enumitem}
\usepackage[portuguese]{babel}


% Title Page
\title{Trabalho III: o método Delta.}
\author{Disciplina: Inferência Estatística \\ Aluna: Iara Cristina Mescua Castro}

\begin{document}
	\maketitle
	
	\textbf{Data de Entrega: 26 de Outubro de 2022.}
	
	\section*{Orientações}
	\begin{itemize}
		\item Enuncie e prove (ou indique onde se pode encontrar a demonstração) de~\underline{todos} os resultados não triviais necessários aos argumentos apresentados;
		\item Lembre-se de adicionar corretamente as referências bibliográficas que utilizar e referenciá-las no texto;
		\item Equações e outras expressões matemáticas também recebem pontuação;
		\item Você pode utilizar figuras, tabelas e diagramas para melhor ilustrar suas respostas;
		\item Indique com precisão os números de versão para quaisquer software ou linguagem de programação que venha a utilizar para responder às questões\footnote{Não precisa detalhar o que foi usado para preparar o documento com a respostas. Recomendo a utilização do ambiente LaTeX, mas fique à vontade para utilizar outras ferramentas.};
	\end{itemize}
	
	
	\section*{Introdução}
	
	Algumas vezes estamos interessados em estimar funções de variáveis aleatórias, em particular funções da média amostral.
	O método Delta permite, sob certas condições, aproximar a distribuição assintótica de funções de variáveis aleatórias.
	Este resultado é extremamente útil em Estatística porque permite obter aproximações sob condições bastante gerais, muitas vezes quando estimadores explícitos não estão disponíveis em forma fechada.
	
	\section*{Questões}
	\begin{enumerate}
		\item Enuncie e prove o método Delta;
		\item Discuta sob quais condições o método funciona e porque;
		\item \textbf{Definição 1: transformações estabilizadoras da variância}.
		Suponha que $E[X_i] = \theta$ é o parâmetro de interesse. 
		O Teorema central do limite diz que
		\begin{equation}
			\sqrt{n}\left(\bar{X}_n - \theta \right) \xrightarrow{d} \textrm{Normal}\left(0, \sigma^2(\theta)\right),
		\end{equation}
		ou seja, a variância da distribuição limite é função de $\theta$.
		Idealmente, gostaríamos\footnote{Por razões que ficarão claras mais à frente no curso.
			Se sua curiosidade não puder esperar, pesquise ``estatística ancilar'' ou ``ancillar statistics''.} que essa distribuição não dependesse de $\theta$.
		Podemos utilizar o método Delta para resolver esse problema.
		Em particular, você demonstrou acima que
		\begin{equation}
			\sqrt{n}\left(g(\bar{X}_n) - g(\theta) \right) \xrightarrow{d} \textrm{Normal}\left(0, \sigma^2(\theta)g^\prime(\theta)^2\right).
		\end{equation}
		O que desejamos então é escolher $g$ de modo que $g^\prime(\theta)\sigma(\theta) = a$ para todo $\theta$.
		Dizemos que $g$ é uma~\textbf{transformação estabilizadora da variância}.
		
		\textbf{Aplicação:} Sejam $X_1, X_2, \ldots, X_n$ uma amostra i.i.d. de uma distribuição normal com média $\mu = 0$ e variância $\sigma^2$,~\textbf{desconhecida}.
		Defina $Z_i = X_i^2$ e $\tau^2 = \operatorname{Var}(Z_i)$.
		\begin{itemize}
			\item[(i)] Mostre que $\tau^2 = 2\sigma^4$.
			\item[(ii)] É possível mostrar que 
			\begin{equation}
				\sqrt{n}\left(\bar{Z}_n - \sigma^2 \right) \xrightarrow{d} \textrm{Normal}\left(0, 2\sigma^4\right).
			\end{equation}
			Proponha uma transformação estabilizadora da variância para este problema\footnote{Note que, como não conhecemos $\sigma^2$, $g$ não pode depender de $\sigma^2$.}
			\textit{Dica}: Encontre $g$ tal que 
			\begin{equation*}
				\sqrt{n}\left(g(\bar{Z}_n) - g(\sigma^2) \right) \xrightarrow{d} \textrm{Normal}\left(0, 2\right).
			\end{equation*}
		\end{itemize}
	\end{enumerate}

	\section*{Respostas}
	\begin{enumerate}
		\item O método Delta deriva a variância de uma função de uma normal com variáveis aleatórias da seguinte forma:
		Suponha que $\bar{X_n}$ tenha uma distribuição normal assintótica, tal que:
		$$\sqrt{n}(\bar{X_n} - \mu) \rightarrow N(0, 
\sigma^2)$$
		
		quando $n \rightarrow \infty$. Seja $g$ é uma função contínua e derivável em $\mu$, onde $g'(\mu) \neq 0$. Então:
		
		$$\sqrt{n}(g(\bar{X_n}) - g(\mu)) \xrightarrow{\text{d}} N(0, (g'(\mu))^2 \sigma^2)$$
		
		\textbf{Prova:}
		A expansão de Taylor permite restaurar uma função diferenciável, g(x), em termos de uma soma infinita das derivadas de g(p).
		
		$$g(x) = g(p) + g'(p)(x-p) + \frac{g''(p)}{2!}(x-p)^2 + \frac{g'''(p)}{3!}(x-p)^3 + ...$$
		
		É possível cortar a expansão após um certo número de termos, para obter uma aproximação de $g(x)$.
		Cortando a expansão após o segundo termo, e substituindo "p" por $\mu$ teremos
		a aproximação de Taylor de primeira ordem de g sobre o ponto $\mu$, e avaliada na variável aleatória $\bar{X_n}$, que é:
		
		$$g(\bar{X_n}) \approx g(\mu) + g'(\mu)(\bar{X_n} - \mu)$$
		
		Subtraindo $g(\mu)$ dos dois lados:
		
		$$g(\bar{X_n}) - g(\mu) \approx g'(\mu)(\bar{X_n} - \mu) $$
		
		Multiplicando por $\sqrt{n}$:
		
		$$\sqrt{n}(g(\bar{X_n}) - g(\mu)) \approx \sqrt{n}(g'(\mu)(\bar{X_n} - \mu)) $$
		
		O Teorema Central do Limite (TCL) diz que:
		$$\sqrt{n}(\bar{X_n} - \mu)) \rightarrow N(0, \sigma^2)$$
		
		
		Então pelo Teorema de Slutsky:
		$$\sqrt{n}(g'(\mu)(\bar{X_n} - \mu)) \xrightarrow{\text{d}} N(0, (g'(\mu))^2 \sigma^2)$$
		
		
		$$\sqrt{n}(g(\bar{X_n}) - g(\mu)) \xrightarrow{\text{d}} N(0, (g'(\mu))^2 \sigma^2)$$
		\newpage
		
		\item Condições: $g$ é derivável e contínuo para a aplicação da expansão de Taylor e $g'(\mu) \neq 0$
		
		É necessário para que:
		$$\frac{\sqrt{n}(g(\bar{X_n}) - g(\mu))}{|g'(\mu)|\sigma} \rightarrow N(0,1)$$
		quando $n \rightarrow \infty$
		
		\item 
		(i)
		$$X \sim (0,\sigma^2)$$
		%$$\sum_{i>1}^n X_i^2 = Gamma\left(\frac{1}{2},\frac{n}{2}\right) = \chi^1_n$$
		$$\frac{X}{\sigma}\sim N(0,1)$$
		Elevando ao quadrado e substituindo $X^2$ por $Z$, temos:
		$$\left(\frac{X}{\sigma}\right)^2 = \frac{Z}{\sigma^2}$$
		Visto que $\frac{X}{\sigma}\sim N(0,1)$, então seu quadrado, $\frac{Z}{\sigma^2}$, tem distribuição qui-quadrado com 1 grau de liberdade:
		$$\frac{Z}{\sigma^2} \sim \chi^2(1)$$
		Sabendo que $Var(\chi^2(m)) = 2m$, então $Var(\chi^2(1)) = 2$:
		$$Var\left(\frac{Z}{\sigma^2}\right) = 2$$
		Pela propriedade: $Var(cX) = c^2 Var(X)$:
		$$Var\left(\frac{Z}{\sigma^2}\right) = \frac{Var(Z)}{\sigma^4} = 2$$
		$$Var(Z) = 2\sigma^4$$
		$$\tau^2 = 2\sigma^4$$
		(ii)
		\begin{equation}
			\sqrt{n}\left(\bar{Z}_n - \sigma^2 \right) \xrightarrow{d} \textrm{Normal}\left(0, 2\sigma^4\right).
		\end{equation}

		Demostramos que:
		
		\begin{equation}
			\sqrt{n}\left(g(\bar{X}_n) - g(\theta) \right) \xrightarrow{d} \textrm{Normal}\left(0, \sigma^2(\theta)g^\prime(\theta)^2\right).
		\end{equation}
		
		Então:
		
		\begin{equation*}
			\sqrt{n}\left(g(\bar{Z}_n) - g(\sigma^2) \right) \xrightarrow{d} \textrm{Normal}\left(0, 2\right).
		\end{equation*}
	
		\begin{equation}
			\sqrt{n}\left(g(\bar{X}_n) - g(\theta) \right) \xrightarrow{d} \textrm{Normal}\left(0, 2\sigma^4(\theta)g^\prime(\theta^2)^2\right).
		\end{equation}
	
		Precisamos resolver a equação diferencial $g'(\sigma^2)$ de forma que:
		
		$$2 = 2\sigma^4 g^\prime(\sigma^2)^2$$
		$$g^\prime(\sigma^2)^2 = \frac{1}{\sigma^4}$$
		$$g^\prime(\sigma^2) = \frac{1}{\sigma^2}$$
		$$g(\sigma^2) = ln(\sigma^2)$$
	\end{enumerate}


	
	% \bibliographystyle{apalike}
	% \bibliography{refs}
	
\end{document}          