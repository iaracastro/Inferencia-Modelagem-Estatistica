\documentclass[article,11pt,a4paper,brazil]{abntex2}

\usepackage{lmodern}
\usepackage[T1]{fontenc}
\usepackage[utf8]{inputenc}
\usepackage{indentfirst}
\usepackage{nomencl}
\usepackage{color}
\usepackage{graphicx}
\usepackage{microtype}
\renewcommand*{\insertchapterspace}{%
	\addtocontents{lof}{\protect\addvspace{10pt}}%
	\addtocontents{lot}{\protect\addvspace{10pt}}}

% ---
% Pacotes fundamentais 
% ---
\usepackage{lmodern}			% Usa a fonte Latin Modern
\usepackage[T1]{fontenc}		% Selecao de codigos de fonte.
\usepackage[utf8]{inputenc}		% Codificacao do documento (conversão automática dos acentos)
\usepackage{indentfirst}		% Indenta o primeiro parágrafo de cada seção.
\usepackage{nomencl} 			% Lista de simbolos
\usepackage{color}				% Controle das cores
\usepackage{graphicx}			% Inclusão de gráficos
\usepackage{microtype} 			% para melhorias de justificação

% Pacotes adicionais, usados apenas no âmbito do Modelo Canônico do abnteX2
% ---
\usepackage{lipsum}				% para geração de dummy text
% ---



% --- Informações de dados para CAPA e FOLHA DE ROSTO ---
\titulo{Modelagem Estatística de [Tema?]}
\tituloestrangeiro{Statistic Modeling of [?]}

\autor{
	Seu nome aqui }

\data{\today}
% ---

% ---
% Configurações de aparência do PDF final

% alterando o aspecto da cor azul
\definecolor{blue}{RGB}{41,5,195}

% informações do PDF
\makeatletter

\hypersetup{
	%pagebackref=true,
	pdftitle={\@title}, 
	pdfauthor={\@author},
	pdfsubject={Modelo de artigo científico com abnTeX2},
	pdfcreator={LaTeX with abnTeX2},
	pdfkeywords={abnt}{latex}{abntex}{abntex2}{atigo científico}, 
	colorlinks=true,       		% false: boxed links; true: colored links
	linkcolor=blue,          	% color of internal links
	citecolor=blue,        		% color of links to bibliography
	filecolor=magenta,      		% color of file links
	urlcolor=blue,
	bookmarksdepth=4
}
\makeatother
% --- 

% ---
% compila o indice
% ---
\makeindex
% ---

% ---
% Altera as margens padrões
% ---
\setlrmarginsandblock{3cm}{3cm}{*}
\setulmarginsandblock{3cm}{3cm}{*}
\checkandfixthelayout
% ---


% --- 
% Espaçamentos entre linhas e parágrafos 
% --- 

% O tamanho do parágrafo é dado por:
\setlength{\parindent}{1.3cm}

% Controle do espaçamento entre um parágrafo e outro:
\setlength{\parskip}{0.2cm}  % tente também \onelineskip

% Espaçamento simples
\SingleSpacing


% ----
% Início do documento
% ----
\begin{document}
	
	% Seleciona o idioma do documento (conforme pacotes do babel)
	%\selectlanguage{english}
	\selectlanguage{brazil}
	
	% Retira espaço extra obsoleto entre as frases.
	\frenchspacing 
	
	% ----------------------------------------------------------
	% ELEMENTOS PRÉ-TEXTUAIS
	% ----------------------------------------------------------
	
	%---
	%
	% Se desejar escrever o artigo em duas colunas, descomente a linha abaixo
	% e a linha com o texto ``FIM DE ARTIGO EM DUAS COLUNAS''.
	% \twocolumn[    		% INICIO DE ARTIGO EM DUAS COLUNAS
	%
	%---
	
	% página de titulo principal (obrigatório)
	\maketitle
	\thispagestyle{empty}
	\begin{citacao}
		Estudo de caso da disciplina de Modelagem Estatística submetido como trabalho da A2 no 5° período de [curso?].\\
		Professor: Luiz Max
	\end{citacao}
	% titulo em outro idioma (opcional)
	
	\newpage
	\thispagestyle{empty}
	% resumo em português
	\begin{resumoumacoluna}
		O tal do abstract. \lipsum[2]
		
		\vspace{\onelineskip}
		
		\noindent
		\textbf{Palavras-chave}: 
	\end{resumoumacoluna}
	
	\newpage	
	\thispagestyle{empty}
	% resumo em inglês
	\renewcommand{\resumoname}{Abstract}
	\begin{resumoumacoluna}
		\begin{otherlanguage*}{english}
			Em inglês. \lipsum[2]
			
			\vspace{\onelineskip}
			
			\noindent
			\textbf{Keywords}: 
		\end{otherlanguage*}  
	\end{resumoumacoluna}
	
	\newpage
	\thispagestyle{empty}
	\tableofcontents*
	\newpage
	
	\section{Introdução}
	\setcounter{page}{1}
	\pagestyle{plain}
	Intro aqui!!
	
	
	\section{Metodologia}
	
	\newpage
	\thispagestyle{plain}
	\section{Resultados}
	
	\lipsum[1]
	
	% ---
	% Finaliza a parte no bookmark do PDF, para que se inicie o bookmark na raiz
	% ---
	
	% ---
	% Conclusão
	% ---
	\section{Discussão}
	
	
	
	\lipsum[3]
	
	% ----------------------------------------------------------
	% ELEMENTOS PÓS-TEXTUAIS
	% ----------------------------------------------------------
	\postextual
	
	% ----------------------------------------------------------
	% Referências bibliográficas
	% ----------------------------------------------------------
	\bibliography{abntex2-modelo-references}
	
	
\end{document}